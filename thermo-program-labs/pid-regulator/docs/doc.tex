\documentclass{article}
\usepackage{graphicx} % Required for inserting images
\usepackage[english, russian]{babel}
\usepackage{listings}
\usepackage{amsmath}
\usepackage{float}
% term
\title{thermo-phys}

\begin{document}

\section{Описание программы регулятор}

\subsection{Пропорциональный регулятор}
Пропорциональная составляющая вырабатывает выходной сигнал, противодействующий отклонению регулируемой величины от заданного значения, наблюдаемого в данный момент времени. Он тем больше, чем больше это отклонение. Если входной сигнал равен заданному значению, то выходной равен нулю.

Однако при использовании только пропорционального регулятора значение регулируемой величины никогда не стабилизируется на заданном значении. Существует так называемая статическая ошибка, которая равна такому отклонению регулируемой величины, которое обеспечивает выходной сигнал, стабилизирующий выходную величину именно на этом значении. Например, в регуляторе температуры выходной сигнал (мощность нагревателя) постепенно уменьшается при приближении температуры к заданной, и система стабилизируется при мощности, равной тепловым потерям. Температура не может достичь заданного значения, так как в этом случае мощность нагревателя станет равна нулю, и он начнёт остывать.

\subsection{Пропорционально-интегральный регулятор}
Интегрирующая составляющая пропорциональна интегралу по времени от отклонения регулируемой величины. Её используют для устранения статической ошибки. Она позволяет регулятору со временем учесть статическую ошибку.

Если система не испытывает внешних возмущений, то через некоторое время регулируемая величина стабилизируется на заданном значении, сигнал пропорциональной составляющей будет равен нулю, а выходной сигнал будет полностью обеспечиваться интегрирующей составляющей.

\subsection{Постановка задачи}

\begin{figure}[H]
    \centering
    \includegraphics{images/scheme.png}
    \caption{схема}
    \label{fig:scheme}
\end{figure}

Имеется тело (\textit{см рис.} \ref{fig:scheme} ) с теплоемкостью $C$ и температурой $T$. Внешняя температура полагается равной 0 ($T_{\text{вн}} = 0 $). Коэффициент теплоотдачи $\alpha$. К телу можно подводить тепло.

Необходимо регулировать подачу тепла таким образом, чтобы поддерживать фиксированную температуру тела.

\subsection{мат. модель}

Исходная задача может быть описана моделью:
\begin{equation}
    \frac{dT}{dt} = P \cdot \frac{T_0}{\tau_0} - \frac{T}{\tau_0}
\end{equation}

У нее можно ввести параметры:
\begin{enumerate}
    \item характерное время остывания (в программе константа TAU\_0)
    \item максимальная температура (в программе TEMP\_0) 
\end{enumerate}

В программе задача решается численно, используется численная модель:

\begin{align}
    T_{i+1} &= T_i + \Delta T \\
    \Delta T &= P \cdot \frac{T_0}{\tau_0} \cdot \Delta t - \frac{T}{\tau_0} \cdot \Delta t
\end{align}

\subsection{файлы}

\subsubsection{plot.py}
plot.py -- файл с функцией plot

функция plot прикрепляет график к окну приложения

\subsubsection{setup\_window.py}
setup\_window.py -- файл с функцией setup\_window

функция setup\_window инициализирует окно и производит расчет показаний виртуального прибора

\subsubsection{регуляторы}
\begin{enumerate}
    \item positional\_regulator -- реализация позиционного регулятора
    \item proportional\_regulator -- реализация пропорционального регулятора
    \item pi\_regulator -- реализация пропорционально-интегрального регулятора
\end{enumerate}

можно использовать любой из реализованных регуляторов, передавая нужный регулятор в качестве аргумента функции \textit{setup\_window} в файле \textit{index.py}

\subsection{Работа программы}
для начала работы необходимо выполнить консольную команду 

\begin{lstlisting}
python index.py
\end{lstlisting}

находясь в папке проекта (\textit{python должен быть установлен и путь до его бинарника должен лежать в PATH})

в результате должно появится окошко на котором будет рисоваться график в реальном времени. снизу от графика написано текущее значение температуры и чекбокс выступающий в роли индикатора включения/выключения мощности.

\includegraphics{images/window.png}

\end{document}

