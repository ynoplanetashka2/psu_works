\documentclass{article}
\usepackage{graphicx} % Required for inserting images
\usepackage[english, russian]{babel}
\usepackage{listings}
\usepackage{amsmath}
\usepackage{float}
% term
\title{thermo-phys}

\begin{document}

\section{Описание программы регулятор}
\subsection{мат. модель}

Исходная задача может быть описана моделью:
\begin{equation}
    \frac{dT}{dt} = P \cdot \frac{T_0}{\tau_0} - \frac{T}{\tau_0}
\end{equation}

У нее можно ввести параметры:
\begin{enumerate}
    \item характерное время остывания (в программе константа TAU\_0)
    \item максимальная температура (в программе TEMP\_0) 
\end{enumerate}

В программе задача решается численно, используется численная модель:

\begin{align}
    T_{i+1} &= T_i + \Delta T \\
    \Delta T &= P \cdot \frac{T_0}{\tau_0} \cdot \Delta t - \frac{T}{\tau_0} \cdot \Delta t
\end{align}

\subsection{файлы}

\subsubsection{plot.py}
plot.py - файл с функцией plot

функция plot прикрепляет график к окну приложения

\subsubsection{setup\_window.py}
setup\_window.py - файл с функцией setup\_window

функция setup\_window инициализирует окно и производит расчет показаний виртуального прибора

\subsubsection{регуляторы}


\subsection{Работа программы}
для начала работы необходимо выполнить консольную команду 


\begin{lstlisting}
py index.py
\end{lstlisting}

находясь в папке проекта

в результате должно появится окошко на котором будет рисоваться график в реальном времени. снизу от графика написано текущее значение температуры и чекбокс выступающий в роли индикатора включения/выключения мощности.

\includegraphics{images/window.png}

\end{document}

